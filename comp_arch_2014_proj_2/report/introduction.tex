\section{Introduction}

In this project, we have implemented a variation of Tomasulo Algorithm for add and multiply instructions with register operands. This alrogithm in used to facilitate out-of-order execution by issuing inctuctions to special buffers called reservation stations, and executing instuctions simultaneously or in the order of true data dependencies rather than the order of issuing.

With the default paramaters provided in the assignment, we have been able to achieve the speedup of over 400\% compared to in-order execution.

\subsection{Sequential version}

In order to verify the results, we have made two implementations of an in-order non-pipelined simulator. One was written from scratch in Python, and another was using the same tomasulo\_sim.o framework. Both ouputted exactly the same results, which we regard as a good evidence that both of them work correctly.

Since by default Python uses arbitrary-precision arithmetic, we have discovered that even in small trace the values in registers go far beyond the capacity of 32-bit integer. Thus, we concluded that these traces highly depend on overflowing and switched to the 32-bit signed integer implementation provided by NumPy library

